%% -----------------------------------
%% ---> A MODIFIER PAR L'ETUDIANT <---
%% -----------------------------------
%%
%% Commandes qui affichent le titre du document, le nom de l'auteur, etc.
\newcommand\monPays{république algérienne démocratique et populaire}
\newcommand\monUniversite{université m'hamed bougara de boumerdes}
\newcommand\monUnivAbbr{UMBB}
\newcommand\monFaculte{faculté des sciences de l'ingénieur}
\newcommand\monFacAbbr{FSI - exINGM}
\newcommand\monTitre{Titre de mon document}
\newcommand\monPrenom{Abdelhak}
\newcommand\monNom{Bougouffa}
\newcommand\monEmail{abougouffa@fedoraproject.org}
%\newcommand\monBinomePrenom{Anis} % Informations de binôme, lesser en commentaire si vous travailler en monôme
%\newcommand\monBinomeNom{Hocine}
%\newcommand\monBinomeEmail{hocine.anis.1992@gmail.com}
\newcommand\monDepartement{InfoTronique}
\newcommand\maDiscipline{systèmes informatiques distribués}
\newcommand\monDiplome{M}        % (M)aster/licence ou (D)octorat
\newcommand\anneeDepot{2017}
\newcommand\moisDepot{juin}
\newcommand\monSexe{M}           % "M" ou "F"
\newcommand\PageGarde{N}         % "O" ou "N"
\newcommand\AnnexesPresentes{O}  % "O" ou "N". Indique si le document comprend des annexes.
\newcommand\mesMotsClef{Liste,de,mot-clés,séparés,par,des,virgules}
%%
%%  DEFINITION DU JURY
%%
%%  Pour la définition du jury, les macros suivantes sont definies:
%%  \PresidentJury, \DirecteurRecherche, \CoDirecteurRecherche, \MembreJury, \MembreExterneJury
%%
%%  Toutes les macros prennent 4 paramètres: Sexe (M/F), Prénom, Nom, Titres
\newcommand\monJury{\PresidentJury{M}{prénom}{nom}{Ph.~D.}\\
\DirecteurRecherche{F}{prénom}{nom}{Ph.~D.}\\
\MembreJury{M}{prénom}{nom}{Prof.}}
